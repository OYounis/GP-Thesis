\documentclass[../main.tex]{subfiles}
\graphicspath{{\subfix{../diagrams/}}}

\begin{document}

\chapter*{
    \begin{center}
        Abstract
    \end{center}}
\addcontentsline{toc}{chapter}{Abstract}
\tab RISC-V Architectures have been widely emerging due to RISC-V being open source and more secure. Booting Linux on a RISC-V core requires implementing RV64-IMFAC[1] the IMFAC stands for Integer Operations, Multiply and Division, Floating Point Operations, Atomic Operation and Compressed Instructions; besides of course having a decent memory hierarchy and virtual memory. Implementing all these features and also expect to tape-out the design in just one year is not realistic. So the challenge is to make the very minimum effort to boot a simple Linux patch on AlexCore[2].\\

We want to launch the design of a new Linux-Capable RISC-V[3] core from Alexandria University. This year we intended to implement the minimum features to boot a simple Linux batch without disturbing the future efforts of scaling to have more features or cores.\cite{Balkind:2016:OOS:2872362.2872414}\cite{21de34cd0dd64073b31ada366dd27796} 

\afterpage{\blankpage}

\end{document}
\documentclass[../main.tex]{subfiles}
\graphicspath{{\subfix{../diagrams/}}}


\begin{document}

\section{Cryptographic IP}
The cryptography IP we used in our SoC was an open source project verilog implementation of the symmetric block cipher AES (NIST FIPS 197), Implemented by Joachim Strömbergson (secworks).\\
\subsection*{Introduction}
This implementation supports 128 and 256 bit keys. The implementation is iterative and process one 128 block at a time. Blocks are processed on a word level with 4 S-boxes in the data path. The S-boxes for encryption are shared with the key expansion and the core can thus not do key update in parallel with block processing.\\
The encipher and decipher block processing datapaths are separated and basically self contained given access to a set of round keys and a block. This makes it possible to hard wire the core to only encipher or decipher operation. This allows the synthesis/build tools to optimize away the other functionality which will reduce the size to about 50\%. This has been tested to verify that decryption is removed and the core still works.\\
For cipher modes such as CTR, CCM, CMAC, GCM the decryption functionality in the AES core will never be used and thus the decipher block processing can be removed.\\
This is a fairly compact implementation. Further reduction could be achived by just having a single S-box. Similarly the performane can be increased by having 8 or even 16 S-boxes which would reduce the number of cycles to two cycles for each round.\\

\subsection*{Core Usage}
Usage sequence:
\begin{itemize}
    \item Load the key to be used by writing to the key register words.
    \item Set the key length by writing to the config register.
    \item Initialize key expansion by writing a one to the init bit in the control register.
    \item Wait for the ready bit in the status register to be cleared and then to be set again. This means that the key expansion has been completed.
    \item Write the cleartext block to the block registers.
    \item Start block processing by writing a one to the next bit in the control register.
    \item Wait for the ready bit in the status register to be cleared and then to be set again. This means that the data block has been processed.
    \item Read out the ciphertext block from the result registers.
\end{itemize}
We implemented this sequence by a finite state machine as our control unit to control the process of the encryption when the custom instruction from the processor asserted.
\subsection*{Implementation results - ASIC}
The core has been implemented in standard cell ASIC processes.\\
\\\textbf{TSMC 180 nm}
Target frequency: 20 MHz Complete flow from RTL to placed gates. Automatic clock gating and scan insertion.\\
When the custom instruction for encryption loaded by the processor, the processor will generate a start signal to start the IP encryption sequence and to stall the processor till the IP finishes the encryption process and assert the done signal, then the result encrypted values will be saved in the CSR register file and the processor will continue the normal operations.
\begin{itemize}
    \item 8 kCells
    \item Aera: 520 x 520 um
    \item Good timing margin with no big cells and buffers.
\end{itemize}

\subsection*{Implementation results - FPGA}
The core has been implemented in Altera and Xilinx FPGA devices.\\
\\\textbf{Altera Cyclone V GX}
\begin{itemize}
    \item 2624 ALMs
    \item 3123 Regs
    \item 96 MHz
    \item 46 cycles/block
\end{itemize}

\\\textbf{Altera Cyclone IV GX}
\begin{itemize}
    \item 7426 LEs
    \item 2994 Regs
    \item 96 MHz fmax
    \item 46 cycles/block
\end{itemize}

\\\textbf{Xilinx Artix 7 200T-3}
\begin{itemize}
    \item 2298 slices
    \item 2989 regs
    \item 97 MHz
    \item 46 cycles/block
\end{itemize}

\end{document}